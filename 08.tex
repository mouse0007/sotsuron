\chapter{拡散性評価}
室内音響学における研究や設計では常に音場の物理的特性と聴感的印象との対応を考えなければならないことから、インパルス応答については直接音領域と初期反射音領域、後期反射音領域と時間軸上で分けて扱うのが一般的である。
\\ 本研究もそれに倣い、観測点位置による影響が大きく及ぶであろう直接音と初期反射音域をとりあげ、考察を行った。
\section{距離減衰特性}
\subsection{初期音に関する物理量}

このとき、$t_A$を最初の反射音が到達するまでの時間、$p_A(t)$を音源からの距離が10mの観測点における直接音の音圧とする。

\begin{equation}
  \label{eq:}
  G_{early} = 10\log_{10}{\frac{\displaystyle\int_0^{80}p^2(t)dt}{\displaystyle\int_0^{t_A}p_{A}^2(t)dt}} \dB
\end{equation}

\begin{equation}
  \label{eq:}
  VG_{early} = 10\log_{10}{\frac{\displaystyle\int_0^{80}p^2(t)cos^2{\delta}dt}{\displaystyle\int_0^{t_A}p_{A}^2(t)dt}} \dB
\end{equation}

\begin{equation}
  \label{eq:}
  LG_{early} = 10\log_{10}{\frac{\displaystyle\int_0^{80}p^2(t)sin^2{\delta}cos^2{\theta}dt}{\displaystyle\int_0^{t_A}p_{A}^2(t)dt}} \dB
\end{equation}

\begin{equation}
  \label{eq:}
  GG_{early} = 10\log_{10}{\frac{\displaystyle\int_0^{80}p^2(t)sin^2{\delta}sin^2{\theta}dt}{\displaystyle\int_0^{t_A}p_{A}^2(t)dt}} \dB
\end{equation}

\subsection{減衰率の算出}
回帰直線の傾きから減衰率[dB/dd]を算出する旨。
\subsection{観測高さの違いと距離減衰特性の分析}
\subsection{客席面吸音率の変化と距離減衰特性の分析}
客席面を吸音性/反射性にしたときに距離減衰特性がどのように変化したかを記述。
\section{反射音方向分布特性}
\subsection{初期音の方向別エネルギ率}

\begin{equation}
  \label{eq:}
  ER_V = 10\log_{10}{\frac{\displaystyle\int_0^{80}p^2(t)cos^2{\delta}dt}{\displaystyle\int_0^{80}p^2(t)dt}} 
\end{equation}

\begin{equation}
  \label{eq:}
  ER_L = 10\log_{10}{\frac{\displaystyle\int_0^{80}p^2(t)sin^2{\delta}cos^2{\theta}dt}{\displaystyle\int_0^{80}p^2(t)dt}} 
\end{equation}

\begin{equation}
  \label{eq:}
  ER_G = 10\log_{10}{\frac{\displaystyle\int_0^{80}p^2(t)sin^2{\delta}sin^2{\theta}dt}{\displaystyle\int_0^{80}p^2(t)dt}} 
\end{equation}

\subsection{DUIの導入}

\begin{equation}
  \label{eq:}
  d_i = \sqrt{\left({ER_V-\frac{1}{3}}\right)^2 + (\left(ER_L-\frac{1}{3}\right)^2 + \left(ER_G-\frac{1}{3}\right)^2} 
\end{equation}

\begin{equation}
    \label{eq:}
    DUI=\frac{\sum_{i=1}^N{d_i}}{N}
\end{equation}

\subsection{観測高さの違いと反射音方向分布特性の分析}

\subsection{客席面吸音率の変化と反射音方向分布特性の分析}