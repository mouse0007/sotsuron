\begin{acknowledgment}
\thispagestyle{fancy}
本論文は筆者が芝浦工業大学工学部建築学科古屋研究室在学中に行った研究をまとめたものです。
\\ この1年間、ご指導ご高配を賜りました芝浦工業大学教授・古屋浩先生に厚くお礼申し上げます。
ゼミでは学生の発言1つ1つに対し、その都度的確かつ細かなご指摘を戴いたほか、研究の進め方から物事に取り組む姿勢、資料の表現方法に至るまで終始厳しくも優しい御指導を賜りました。また、研究に関する御指導だけでなく、日頃の研究室活動においても常に温かいご配慮を賜りました。
研究室配属当初からの筆者の希望であった「工学的に建築を捉える」というテーマに向かって、
これまで研究指導して下さいましたことに重ねて御礼申し上げます。
\\ 古屋研究室の先輩である藤田鋭志さん、原彩乃さん、平舘勇馬さんには研究の具体的なアドバイスから資料の作り方に至るまで多くの御助言をいただきました。ここに記して感謝申し上げます。
\\ 苦楽を共にした同期の黒瀬彰君、清水大世君、染野紗結美さん、田代知樹君、中田絢乃さん、堀野量子さんには公私にわたり本当にお世話になりました。同期の皆様には研究の相談に乗っていただいただけでなく、日々の活動においても多大な御協力を戴きました。心より感謝申し上げます。
\\ 最後に、私がこうして卒業研究に取り組めるよう、いつも支援して下さる両親へ深く感謝いたします。
\\
\begin{flushright}
2019年1月29日\\
松崎 広夢
\end{flushright}
\end{acknowledgment}
