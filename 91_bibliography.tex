\begin{bib}[100]
\thispagestyle{fancy}

% \bibitem{参照用名称}
%   著者名:
%   \newblock 文献名,
%   \newblock 書誌情報,出版年.

%_01_tex
\bibitem{01-1}
  ハインリッヒ・クットルフ:
  \newblock 室内音響学 -建築の響きとその理論-
  \newblock \\ 市ケ谷出版社, p.276 (2003)

\bibitem{01-2}
  Leo L. Beranek : 
  \newblock Analysis of Sabine and Eyring equations and their  application to concert hall audience and chair absorption.
  \newblock \\ J. acoust. Soc. Am., 120, 3, pp.1399-1410 (2006)

\bibitem{01-3}
  前川純一, 森本政之, 坂上公博 : 
  \newblock 建築・環境音響学第3版
  \newblock \\ 共立出版 (2011.9)
  
\bibitem{02-1}
  レオ・L.ベラネク, 日高孝之, 永田穂 : 
  \newblock コンサートホールとオペラハウス 音楽と空間の響きと建築
  \newblock \\ シュプリンガー・フェアラーク東京 (2005.11)

\bibitem{02-2}
  日本騒音制御工学会編 : 
  \newblock 騒音制御工学ハンドブック
  \newblock \\ 技報堂 (2001)

\bibitem{02-4}
  小林茂雄, 中島裕輔, 西村直也, 古屋浩, 吉永美香 : 
  \newblock はじめての建築環境工学
  \newblock \\彰国社(2014.9)

\bibitem{02-3}
  CATT : 
  \newblock CATT-Acoustic v8.0 User's Manual
  \newblock \\(2002.4)

\bibitem{02-5}
  尾本章 : 
  \newblock 1.幾何音響学の考え方, 特集 : されど幾何音響シミュレーション
  \newblock \\音響技術, No.129(vol.34 no.1), pp.2-7, (2005.3)

\bibitem{02-6}
  石田康二 : 
  \newblock 3.幾何音響学に基づく各種シミュレーション手法について, 特集 : されど幾何音響シミュレーション
  \newblock \\音響技術, No.129(vol.34 no.1), pp.14-23, (2005.3)

 \end{bib}