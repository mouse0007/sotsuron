\chapter{評価手法}
\section{一元配置分散分析}
距離に依存しない指標の分析

\section{共分散分析}
距離に依存する指標の分析

\section{室内音響物理指標の弁別閾}
水準間に有意な差があったとき、その差は人間が知覚できるほどの変化量であるかどうかをJND ( just noticeable difference : 感覚上ちょうど感知できる物理刺激の最小変化量 ) を用いて判断する。
ISO3382-1$^{\text{\cite{iso3382}}}$に整理されている室内音響物理指標の弁別閾(JND)を\tabref{jnd}に示す。

\begin{table}[htbp]
\centering
\caption{\hspace{1mm}JND of each acoustic parameters}
\label{tab:jnd}
\begin{tabular}{lccc}
\Hline
\multicolumn{1}{c}{Subjective listener aspect} & Acoustic quantity & \begin{tabular}[c]{@{}c@{}}Single number\\ freq. ave.$^*${[}Hz{]}\end{tabular} & \begin{tabular}[c]{@{}c@{}}Just noticeable difference\\ (JND)\end{tabular} \\ \hline
Subjective level of sound & $G${[}dB{]} & 500 to 1000 & 1[dB] \\
Perceived reverberance & $EDT${[}s{]} & 500 to 1000 & Rel.5[\%] \\
Perceived clarity of sound & $C_{80}${[}dB{]} & 500 to 1000 & 1[dB] \\
 & $D_{50}${[}\%{]} & 500 to 1000 & 5{[}\%{]} \\
Apparent source width (ASW) & $LF${[}\%{]} & 500 to 1000 & 5{[}\%{]} \\
\Hline
\multicolumn{4}{l}{$^{*}$ : The single number frequency averaging denotes the arithmetical average for the octave bands.}
\end{tabular}
\end{table}