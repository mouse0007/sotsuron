\chapter{結論}
評価対象領域を拡張することにより、オーディトリウム空間全体の音場特性を拡散性の観点から考察した。その結果、座席面遠方音場は近傍に比べ音響エネルギ分布及び反射音到来方向分布が一様であり、より拡散した音場であることがわかった。これらの結果は「オーディトリウムの“底”で演奏を聴く」という旧態依然とした鑑賞形態に対し座席面遠方音場での鑑賞が最適音場の実現に寄与することを示唆しており、新たなホール空間を設計するための可能性を示すものである。